\documentclass[11pt, a4paper]{article}

\usepackage[utf8]{inputenc}
\usepackage[T1]{fontenc}
\usepackage[margin=1.0in]{geometry}
\usepackage[british]{babel}
\usepackage{indentfirst}
\usepackage{array,booktabs,longtable}
\usepackage{multirow}
\usepackage{listings}

\usepackage{comment}

\newcommand{\prog}[1]{{\tt#1}}
\newcommand{\underscore}{$\_\,$}

\begin{document}

\section{Abstract}
In this contribution, we investigate the problem of "SwitchBoxes".
The goal is, given $n$ wires, to generate all the permutations of these wires by using "boxes", that swap two wires. We tried to minimize the number of boxes that were necessary.
We introduce a conjecture that puts our problem in relation with binary insertion sort. We successfully used a heavily optimized algorithm to prove this conjecture for small values of $n$.

\section{Introduction}
The SwitchBoxes problem is a combinatory problem. Given $n$ wires, (coming from the top), we try to generate all the permutations of these wires by using boxes. A box takes two wires, and is allowed to swap these wires.



\end{document}
