\documentclass[11pt, a4paper]{article}

\usepackage[utf8]{inputenc}
\usepackage[T1]{fontenc}
\usepackage[margin=1.0in]{geometry}
\usepackage[british]{babel}
\usepackage{indentfirst}
\usepackage{array,booktabs,longtable}
\usepackage{multirow}
\usepackage{listings}
\usepackage{amssymb}
\usepackage{amsmath}
\usepackage{amsthm}
\usepackage{url}
\usepackage{graphicx}
\usepackage{caption}
\usepackage{subcaption}
\usepackage{mathtools}
\usepackage{algpseudocode}
\usepackage{algorithm}

\newcommand{\includesvg}[1]{%
  \input{#1.pdf_tex}%
}

\graphicspath{{svg/}}

\DeclarePairedDelimiter{\ceil}{\lceil}{\rceil}
\DeclarePairedDelimiter{\floor}{\lfloor}{\rfloor}

\usepackage{comment}

\newtheorem*{Conj*}{Conjecture}

\newcommand{\prog}[1]{{\tt#1}}
\newcommand{\underscore}{$\_\,$}

\begin{document}

\title{ Paper without a name }
\author{A. Auvolat \and A. Fromherz \and N. Jeannerod }
\date{June 01, 2014}
\maketitle

\begin{abstract}

In this contribution, we investigate the problem of ``SwitchBoxes''.  The goal is,
given $n$ wires, to generate all the permutations of these wires by using
``boxes'', that swap two wires. We tried to minimize the number of boxes that were
necessary.  We introduce a conjecture that puts our problem in relation with
binary insertion sort.  We successfully used a heavily optimized algorithm to
prove this conjecture for small values of $n$.

\end{abstract}

\section{Introduction}

The SwitchBoxes problem is a combinatory problem. Given $n$ wires, we try to
generate all the permutations of these wires by using boxes.  A box takes two
wires and a control bit, and swaps the two entry wires iff the control bit is
set. We can concatenate these boxes to get a configuration of boxes.  The
configuration is said valid if, by giving the set of boxes different
configurations (ie. swapping or not the wires on every box), we can obtain any
permutation of the $n$ wires.

Mathematically, we consider that the wires are numbers from 1 to $n$. A box is a
permutation $\tau(\epsilon) = (i,j)^{\epsilon}$, where $\epsilon$ represents the
control bit for the box and is either 0 or 1.  A configuration of boxes $C$ is
the concatenation of $k$ boxes $\tau_1, \ldots, \tau_k$, that means,
$C(\epsilon_1, \ldots, \epsilon_k) =
\tau_1^{\epsilon_1}\circ\ldots\circ\tau_k^{\epsilon_k}$.  The configuration $C$
is said valid if $C(\{0,1\}^k)= \mathfrak{S}_n$.

Given this problem, we tried to determine the best lower bound of the number of
boxes $k$ which is valid for $\mathfrak{S}_n$, ie generates all the permutations of $n$
wires.

We can easily determine trivial lower and upper bounds of the number of
boxes~:\\ Given the fact that $C$ is an application, we need to have
$Card(\{0;1\}^{k}) \geq Card(\mathfrak{S}_n)$. We immediately obtain that $k$
has to be greater than $\log_2(n!)$.\\ We also have an upper bound~: The
traditional bubblesort gives a valid configuration of the SwitchBoxes problem. We
obtain that the optimal $k$ is smaller than $\frac{(n-1)(n-2)}{2}$.

In this contribution, we first recall how the sequence of the maximal number of
comparisons for sorting $n$ elements by binary insertion is defined and develop
how the optimal $k$ seems to be linked to it.  We then explain why we guess that
those numbers are equal.

In a second step, we show how we proved this conjecture for $n$ from 2 to 6, and
how we demonstrated that the number of comparisons is an upper bound of the
optimal number of boxes for $n$ from 7 to 13 by optimizing the checking
algorithm and by constructing a precise configuration.

In addition, we give some ideas we tried to develop in order to prove
mathematically this conjecture.

In the rest of this paper, $opt_n$ will be the optimal number of boxes for $n$
wires.


\section{The A001855 sequence}

In this paragraph, we will develop on the A001855 sequence, here denoted $u_n$,
which seems to be in relation with our problem.

$u_n$ is defined as the maximal number of comparisons done by a binary
insertion sort running on a table of $n$ items.

\begin{algorithm}
\begin{algorithmic}
\Function{FindPos}{$n, T, b, e$}
	\Comment{Find $n$ in $T[b, b+1, \dots, e-1]$}
	\State \textbf{assert} $b < e$
	\If {$b + 1 = e$}
		\State \Return $b$
	\Else
		\State $m \gets \floor{(b+e)/2}$
		\If {$T[m] > n$}
			\State \Return $\mathrm{FindPos}(n, T, b, m)$
		\Else
			\State \Return $\mathrm{FindPos}(n, T, m+1, e)$
		\EndIf
	\EndIf
\EndFunction
	\\
\Function{Sort}{$T$}
	\State $n \gets \mathrm{size}(T)$
	\State $S \gets [ T[0] ]$
	\For{$i = 1$ \textbf{to} $n-1$}
		\State $p \gets \mathrm{FindPos}(T[i], S, 0, \mathrm{size}(S))$
		\State \textbf{insert} $T[i]$ \textbf{in} $S$ \textbf{at} $p$
	\EndFor
	\State \Return $S$
\EndFunction
\end{algorithmic}
\caption{Binary insertion sort algorithm}
\end{algorithm}

This algorithm progressively builds $S$, the table of the elements of $T$ in
increasing order. This is done by progressively inserting the elements of $T$ in
$S$ at a position calculated by the $FindPos$ function. The $FindPos$ function
is based on a binary search, therefore does $O(log(S.size))$ comparisons.

The $u_n$ suite is defined as follows :

$$u_1 = 0$$
$$\forall n \in \mathbb{N}^*, u_{n+1} = u_n + \ceil{log_2(n)}$$

This sequence is described on the OEIS\footnote{Online Encyclopedia of Integer
Sequence}, and is referenced under the name A001855.

This sequence is linked to the construction of complete binary trees : $u_n$ is the sum of the depths of all nodes in the complete binary tree containing $n-1$ nodes (see \cite{btree}).


\section{Our conjecture}

To understand the problem, we first tried manually to get $opt_n$ for very small
values of $n$.\\
For $n=2$, the answer is quite simple~: One box is enough and necessary to swap
the two wires.\\

\begin{figure}
 \centering
 \begin{minipage}{.5\textwidth}
    \centering
     \def\svgwidth{0.3\textwidth}
     \includesvg{svg/boxsys-2-1}
     \caption{Trivial solution for 2 wires}
 \end{minipage}%
 \begin{minipage}{.5\textwidth}
    \centering
     \def\svgwidth{0.5\textwidth}
     \includesvg{svg/boxsys-4-6}
     \caption{Simple non-optimal solution for 4 wires}
 \end{minipage}
\end{figure}

For $n=3$, two boxes aren't enough. Using a permutation, we can consider that
the first box swap the wires 1 and 2, and that the second box swap the wires 2
and 3.  Then the permutation $(1,3)$ can't be generated with this configuration.
Hence, we have $opt_3=3$.


\begin{figure}
    \centering
    \def\svgwidth{0.7\textwidth}
    \includesvg{svg/boxsys-8-20}
    \caption{Simple non-optimal solution for 8 wires}
\end{figure}

On these representations, a box is a horizontal line that can swap the wires the points are on.

For greater values of $n$, we implemented a naive and brutal algorithm that for
$k$ given, creates all the possible configurations of $k$ boxes, and checks if
one of these configurations is valid. We first try it with $k = opt_{n-1}+1$. Then,
we increment $k$ and try again until we get a valid configuration.  With this
algorithm, we easily get $opt_n$.\\
However, this only works for small values of $n$. For $n$ greater than 6, we
lack memory and time to achieve the computation.

Using this algorithm, we got the results in the table of figure~\ref{box-count}.

\begin{figure}
    \centering
    \begin{tabular}{|c|c|}
    \hline
    Number of wires $n$ & $opt_n$ \\
    \hline
    2 & 1 \\
    3 & 3 \\
    4 & 5 \\
    5 & 8 \\
    6 & 11 \\
    \hline
    \end{tabular}
    \caption{Minimum box count for small number of wires, checked experimentally}
    \label{box-count}
\end{figure}

According to the fact that this sequence is exactly the beginning of the $u_n$
sequence described in the previous paragraph, our conjecture is that the
sequences are equal.

\begin{Conj*}
$opt_n$ is equal to the maximal number of comparisons for sorting $n$ elements
by binary insertion.
\end{Conj*}


In order to prove this conjecture for greater values, we tried to optimize our
checking algorithm. That will be described in the next part.

\subsection{First examples}

$n=2$ : $C_2(e_1) = (1,2)^{e_1}$

$n=3$ : $C_3(e_1,e_2,e_3) = (1,2)^{e_1} \circ (2,3)^{e_2} \circ (1,2)^{e_3}$

$n=4$ : $C_4(e_1,e_2,e_3,e_4,e_5) = (1, 2)^{e_1} \circ (3,4)^{e_2} \circ
(1,3)^{e_3} \circ (2, 4)^{e_4} \circ (1,2)^{e_5}$

\subsection {Notation}

We abbreviate the previous notation in the following way :

$$C_2 = (1,2)$$

$$C_3 = (1,2)(2,3)(1,2)$$

$$C_4 = (1,2)(3,4)(1,3)(2,4)(1,2)$$

\section{Hypothesis box system}

We have found a box system that seems to generate all permutations for all $n$.
We have managed to prove that this box system is valid for small values of $n$
($n \leq 13$), and have found that the box systems were still valid when some
boxes were removed.

\subsection{The basis box configuration}

We first define the $q$-block at position $i$ by :

$$B(q,i) = (q,q+2^i)(q+1,q+1+2^i)\cdots(q+2^i-1)(q+2*2^i-1)$$

For example: 
$$B(0,i) = (i,i+1)$$
$$B(1,i) = (i, i+2)(i+1, i+3)$$
$$B(2, i) = (i,i+4)(i+1,i+5)(i+2,i+6)(i+3,i+7)$$

Then, for $n=2^p$ we define for $q < p$ the $(p,q)$-line by :

$$L(p,q) = B(q,0) B(q,2^{q+1}) \cdots B(q,2^p-2^{q+1})$$

For example:
$$L(3,0) = (0,1)(2,3)(4,5)(6,7)$$
$$L(3,1) = (0,2)(1,3)(4,6)(5,7)$$
$$L(3,2) = (0,4)(1,5)(2,6)(3,7)$$

We now conjecture that the following configuration generates all permutations
for $n=2^p$ :

$$C_{2^p}^0 = L(p,0) L(p,1) L(p,2) \cdots L(p,p-1) L(p,p-2) \cdots L(p,1) L(p,0)$$

For example :

$$
\begin{aligned}
    C_8^0 & = & L(3,0) L(3,1) L(3,2) L(3,1) L(3,0) \\
        & = & (0,1)(2,3)(4,5)(6,7) \\
        & & (0,2)(1,3)(4,6)(5,7) \\
        & & (0,4)(1,5)(2,6)(3,7) \\
        & & (0,2)(1,3)(4,6)(5,7) \\
        & & (0,1)(2,3)(4,5)(6,7)
\end{aligned}$$

\subsection{Truncation}

For $n$ which is not a power of two, we construct $C_n^0$ by taking
$C_{2^{log_2(n)}}^0$ and keeping only boxes $(i, j)$ having $i, j < n$.

For example, $C_5^0$ can be generated from $C_8^0$ and by removing quite a lot
of the boxes. We obtain the following solution :

$$C_5^0 = (0,1)(2,3)(0,2)(1,3)(0,4)(0,2)(1,3)(0,1)(2,3)$$

\begin{figure}
    \centering
    \begin{minipage}{.5\textwidth}
        \centering
        \def\svgwidth{0.7\textwidth}
        \includesvg{svg/boxsys-5-9}
        \caption{Solution for 5 wires}
    \end{minipage}%
    \begin{minipage}{.5\textwidth}
        \centering
        \def\svgwidth{0.85\textwidth}
        \includesvg{svg/boxsys-6-12}
        \caption{Solution for 6 wires}
    \end{minipage}
\end{figure}

\subsection{Testing the box system}

For a box system $S = b_1 b_2 \dots b_p$, we construct the sets :

$$Perm_0 = { id }$$
$$Perm_k = (b_1 b_2 \dots b_k)(\left\{0,1\right\}^k)$$

The permutation set $Perm_p$ therefore contains all the possible permutations we
can generate with the $p$ boxes. If $Perm_p = \mathfrak{S}_n$ then the box system
is valid.

Each set $Perm_k$ is represented in memory as a bitset containing $n!$ bits, one
for each permutation of $\mathfrak{S}_n$. The isomorphism between the
permutations of $\mathfrak{S}_n$ and the numbers $\left\{0,1,\dots,n!-1\right\}$ is
constructed using the standard method described in (a citation).

The representation as a bitset is the most compact representation we have found.

\subsection{Useless box elimination}

We can easily detect that $Perm_{k+1} = Perm_k$ when $ \# Perm_{k+1} = \# Perm_k$,
because $Perm_k \subset Perm_{k+1}$. When this condition is fulfilled, then we
know that the box $b_{k+1}$ is useless because it adds no permutation to the set
of generated permutations. We can therefore remove it from the box system.

By removing the boxes that this method detects as useless from the box system we
constructed in the previous paragraphs, we have constructed box systems that
generate all the permutations for $n \leq 13$ and that have $u_n$ boxes (ie the
conjectured optimum). We have not managed to run the test for $n \geq 14$,
because the required RAM exceeds 20Go, which is not a resource we have in our
possession.

\section{Conclusion}

All these results seem to confirm our intuition.
To go further, we unsuccessfully tried to find applications from the set of trees representing a sort by binary insertion to the set of valid configurations of boxes 
that would be either one to one or onto, in order to find an inequality between these two sequences of integers.
We also tried to put our construction of configurations in relation with binary insertion sort. 

\begin{thebibliography}{1}

\bibitem{btree}Sung-Hyuk Cha, {\em On Integer Sequences Derived from Balanced k-ary Trees}, Applied Mathematics in Electrical and Computer Engineering, 2012.

\end{thebibliography}

\end{document}
