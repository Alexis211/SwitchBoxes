\documentclass[11pt, a4paper]{article}

\usepackage[utf8]{inputenc}
\usepackage[T1]{fontenc}
\usepackage[margin=1.0in]{geometry}
\usepackage[british]{babel}
\usepackage{indentfirst}
\usepackage{array,booktabs,longtable}
\usepackage{multirow}
\usepackage{listings}
\usepackage{amssymb}
\usepackage{amsmath}
\usepackage{amsthm}
\usepackage{url}
\usepackage{graphicx}
\usepackage{caption}
\usepackage{subcaption}

\newcommand{\includesvg}[1]{%
  \input{#1.pdf_tex}%
}

\graphicspath{{svg/}}


\usepackage{comment}

\newtheorem*{Conj*}{Conjecture}

\newcommand{\prog}[1]{{\tt#1}}
\newcommand{\underscore}{$\_\,$}

\begin{document}

\title{Paper without a name}
\author{A. Auvolat \and A. Fromherz \and N. Jeannerod}
\date{June 01, 2014}
\maketitle

\begin{abstract}

In this contribution, we investigate the problem of "SwitchBoxes".  The goal is,
given $n$ wires, to generate all the permutations of these wires by using
"boxes", that swap two wires. We tried to minimize the number of boxes that were
necessary.  We introduce a conjecture that puts our problem in relation with
binary insertion sort.  We successfully used a heavily optimized algorithm to
prove this conjecture for small values of $n$.

\end{abstract}

\section{Introduction}

The SwitchBoxes problem is a combinatory problem. Given $n$ wires, (coming from
the top), we try to generate all the permutations of these wires by using boxes.
A box takes two wires, and is allowed to swap these wires. We can concatenate
these boxes to get a configuration of boxes.  The configuration is said valid
if, by swapping or not the wires on every box, we can obtain any permutation of
	the $n$ wires.

Mathematically, we consider that the wires are numbers from 1 to $n$. A box is a
permutation $\tau(\epsilon) = (i,j)^{\epsilon}$, where $\epsilon$ is equal to 0
or 1.  A configuration of boxes C is the concatenation of k boxes $\tau_1,
\ldots, \tau_k$, that means, $C(\epsilon_1, \ldots, \epsilon_k) =
\tau_1^{\epsilon_1}\circ\ldots\circ\tau_k^{\epsilon_k}$.  The configuration C is
said valid if $C(\{0,1\}^k)= \mathfrak{S}_n$.

Given this problem, we tried to determine the best lower bound of the number of
boxes $k$.

We can easily determine trivial lower and upper bounds of the number of
boxes~:\\ Given the fact that C is an application, we need to have
$Card(\{0;1\}^{k}) \geq Card(\mathfrak{S}_n)$. We immediateley obtain that $k$
has to be greater than $\log_2(n!)$.\\ We also have an upper bound~: The
traditionnal bubblesort is a valid configuration of the SwitchBoxes problem. We
obtain that the optimal k is smaller than $\frac{(n-1)(n-2)}{2}$.

In this contribution, we first develop how the optimal k seems to be linked to
the maximal number of comparisons for sorting n elements by binary insertion.
We explain why we guess that those numbers are equal.

In a second step, we show how we proved this conjecture for $n$ from 2 to 6, and
how we demonstrated that the number of comparisons is an upper bound of the
optimal number of boxes for $n$ from 7 to 13 by optimizing the checking
algorithm and by constructing a precise configuration.

In addition, we give some ideas we tried to develop in order to prove
mathematically this conjecture.

In the rest of this paper, $opt_n$ will be the optimal number of boxes for $n$
wires.

% Part 0. Introduction
% Part 1. Conjecture
% Part 2. Optimisation
% Part 3. Pistes de recherche

\begin{figure}
 \centering
 \vspace{+20pt}
 \def\svgwidth{0.15\textwidth}
 \includesvg{svg/2wires_box}
 \caption{Machin truc thingy}
\end{figure}

\begin{figure}
 \centering
 \vspace{+20pt}
 \def\svgwidth{0.25\textwidth}
 \includesvg{svg/4wires_box}
 \caption{More stuff}
\end{figure}


\section{Our conjecture}

To understand the problem, we first tried manually to get $opt_n$ for very small
values of $n$.\\
For $n=2$, the answer is quite simple~: One box is enough and necessary to swap
the two wires.\\
For $n=3$, two boxes aren't enough. Using a permutation, we can consider that
the first box swap the wires 1 and 2, and that the second box swap the wires 2
and 3.  Then the permutation $(1,3)$ can't be generated with this configuration.
Hence, we have $opt_3=3$.

For greater values of $n$, we implemented a naive and brutal algorithm that for
$k$ given, creates all the possible configurations of $k$ boxes, and checks if
one of these configurations is valid. We first try it with $opt_{n-1}+1$. Then,
we increment $k$ and try again until we get a valid configuration.  With this
algorithm, we easily get $opt_n$.\\
However, this only works for small values of $n$. For $n$ greater than 6, we
lack memory and time to achieve the computation.

Using this algorithm, we got the following results~:

\begin{tabular}{|c|c|}
\hline
Number of wires $n$ & $opt_n$ \\
\hline
2 & 1 \\
3 & 3 \\
4 & 5 \\
5 & 8 \\
6 & 11 \\
\hline
\end{tabular}\\

According to the fact that this sequence is exactly the beginning of the
sequence A001855 of the Encyclopedia of Integer Sequences\footnote{The On-Line
Encyclopedia of Integer Sequences, Sorting numbers: maximal number of
comparisons for sorting n elements by binary insertion,
\url{http://oeis.org/A001855}}, corresponding to the maximal number of
comparisons for sorting $n$ elements by binary insertion, we guessed that these
sequences were equal.

\Large
\begin{Conj*}
$opt_n$ is equal to the maximal number of comparisons for sorting $n$ elements
by binary insertion.
\end{Conj*}
\normalsize

In order to prove this conjecture for greater values, we tried to optimize our
checking algorithm. That will be described in the next part.

\subsection{First examples}

$n=2$ : $S_2(e_1) = (1,2)^{e_1}$

$n=3$ : $S_3(e_1,e_2,e_3) = (1,2)^{e_1} \circ (2,3)^{e_2} \circ (1,2)^{e_3}$

$n=4$ : fill in later

\subsection {Notation}

We abbreviate the previous notation in the following way :

$$S_2 = (1,2)$$

$$S_3 = (1,2)(2,3)(1,2)$$

$$S_4 = (1,2)(3,4)(1,3)(2,4)(1,2)$$

\section{Hypothesis box system}

We have found a box system that seems to generate all permutations for all $n$.
We have managed to prove that this box system is valid for small values of $n$
($n \leq 13$), and have found that the box systems were still valid when some
boxes were removed.

\subsection{The basis box configuration}

We first define the $q$-block at position $i$ by :

$$B(q,i) = (q,q+2^i)(q+1,q+1+2^i)\cdots(q+2^i-1)(q+2*2^i-1)$$

For example: 
$$B(0,i) = (i,i+1)$$
$$B(1,i) = (i, i+2)(i+1, i+3)$$
$$B(2, i) = (i,i+4)(i+1,i+5)(i+2,i+6)(i+3,i+7)$$

Then, for $n=2^p$ we define for $q < p$ the $(p,q)$-line by :

$$L(p,q) = B(q,0) B(q,2^{q+1}) \cdots B(q,2^p-2^{q+1})$$

For example:
$$L(3,0) = (0,1)(2,3)(4,5)(6,7)$$
$$L(3,1) = (0,2)(1,3)(4,6)(5,7)$$
$$L(3,2) = (0,4)(1,5)(2,6)(3,7)$$

We now conjecture that the following configuration generates all permutations
for $n=2^p$ :

$$S_{2^p}^0 = L(p,0) L(p,1) L(p,2) \cdots L(p,p-1) L(p,p-2) \cdots L(p,1) L(p,0)$$

For example :

$$
\begin{aligned}
	S_8^0 & = & L(3,0) L(3,1) L(3,2) L(3,1) L(3,0) \\
        & = & (0,1)(2,3)(4,5)(6,7) \\
		& & (0,2)(1,3)(4,6)(5,7) \\
		& & (0,4)(1,5)(2,6)(3,7) \\
		& & (0,2)(1,3)(4,6)(5,7) \\
        & & (0,1)(2,3)(4,5)(6,7)
\end{aligned}$$

\subsection{Truncation}

For $n$ which is not a power of two, we construct $S_n^0$ by taking
$S_{2^{log_2(n)}}^0$ and keeping only boxes $(i, j)$ having $i, j < n$.

For example, $S_5^0$ can be generated from $S_8^0$ and by removing quite a lot
of the boxes. We obtain the following solution :

$$S_5^0 = (0,1)(2,3)(0,2)(1,3)(0,4)(0,2)(1,3)(0,1)(2,3)$$

\subsection{Testing the box system}

\subsection{Useless box elimination}




\end{document}
