\documentclass[11pt, a4paper]{article}

\usepackage[utf8]{inputenc}
\usepackage[T1]{fontenc}
\usepackage[margin=1.0in]{geometry}
\usepackage[british]{babel}
\usepackage{indentfirst}
\usepackage{array,booktabs,longtable}
\usepackage{multirow}
\usepackage{listings}
\usepackage{amssymb}
\usepackage{amsmath}

\usepackage{comment}

\newcommand{\prog}[1]{{\tt#1}}
\newcommand{\underscore}{$\_\,$}

\begin{document}

\section{Abstract}
In this contribution, we investigate the problem of "SwitchBoxes".
The goal is, given $n$ wires, to generate all the permutations of these wires by using "boxes",
 that swap two wires. We tried to minimize the number of boxes that were necessary.
We introduce a conjecture that puts our problem in relation with binary insertion sort. 
We successfully used a heavily optimized algorithm to prove this conjecture for small values of $n$.

\section{Introduction}
The SwitchBoxes problem is a combinatory problem. Given $n$ wires, (coming from the top), 
we try to generate all the permutations of these wires by using boxes. 
A box takes two wires, and is allowed to swap these wires. We can concatenate these boxes to get a configuration of boxes.
The configuration is said valid if, by swapping or not the wires on every box, we can obtain any permutation of the $n$ wires.

Mathematically, we consider that the wires are numbers from 1 to $n$. A box is a permutation $\tau(\epsilon) = (i,j)^{\epsilon}$, where $\epsilon$ is equal to 0 or 1.
A configuration of boxes C is the concatenation of k boxes $\tau_1, \ldots, \tau_k$, that means, $C(\epsilon_1, \ldots, \epsilon_k) = \tau_1^{\epsilon_1}\circ\ldots\circ\tau_k^{\epsilon_k}$.
The configuration C is said valid if $C(\{0,1\}^k)= \mathfrak{S}_n$.

Given this problem, we tried to determine the best lower bound of the number of boxes $k$.

We can easily determine trivial lower and upper bounds of the number of boxes~:\\
Given the fact that C is an application, we need to have $Card(\{0;1\}^{k}) \geq Card(\mathfrak{S}_n)$. We immediateley obtain that $k$ has to be greater than $\log_2(n!)$.\\
We also have an upper bound~:
The traditionnal bubblesort is a valid configuration of the SwitchBoxes problem. We obtain that the optimal k is smaller than $\frac{(n-1)(n-2)}{2}$.

In this contribution, we first develop how the optimal k seems to be linked to the maximal number of comparisons for sorting n elements by binary insertion.
We explain why we guess that those numbers are equal.

In a second step, we show how we proved this conjecture for $n$ from 1 to 6, and how we demonstrated that the number of comparisons is an upper bound of the optimal number of boxes 
for $n$ from 7 to 13 by optimizing the checking algorithm and by constructing a precise configuration.

In addition, we give some ideas we tried to develop in order to prove mathematically this conjecture.

In the rest of this paper, $opt_n$ will be the optimal number of boxes for $n$ wires.

% Part 0. Introduction
% Part 1. Conjecture
% Part 2. Optimisation
% Part 3. Pistes de recherche


\section{Our conjecture}


\section{Hypothesis box system}

\subsection{The basis set}

\subsection{Truncation}

\subsection{Testing the box system}

\subsection{Useless box elimination}




\end{document}
